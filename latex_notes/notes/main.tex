% !TEX program = xelatex
\documentclass[a4paper, 10pt]{book}
\usepackage{amsmath}
\usepackage{amssymb}
\usepackage{ctex}
\usepackage{sectsty}
\usepackage{graphicx}
\chapternumberfont{\tiny}
% for theorem definition, one could use \begin{theorem} \end{theorem} and so on
\newtheorem {theorem} {Theorem} [section]
\newtheorem {proof} {Proof} [section]
% abbr definition
\newcommand {\Q}{\mathbb{Q}}
\newcommand {\Z}{\mathbb{Z}}
\newcommand {\idealO}{\mathcal{O}}
\newcommand {\idealo}{\text{\scriptsize{$\mathcal{O}$}}}

\begin{document}
\chapter{基础代数内容}
中文中文

\chapter{Neukrich: Algebraic Number Theory}
\section{整扩张}
给定可分域扩张L|K的一组基底$\alpha_1, \alpha_2, \cdots, \alpha_n$,其判别式定义为
$$
d(\alpha_1, \cdots, \alpha_n) = \det((\sigma_i \alpha_j))^2
$$
其中$\sigma_i$取遍$L\to \bar{K}$的K-嵌入(i.e.$L$到K的代数闭包的所有固定$K$的同构)。对于形如$1, \theta, \cdots, \theta^{n-1}$这样的基底,其判别式为

\centerline{\scalebox{0.7}{$
\begin{aligned}
    d(1, \theta, \cdots, \theta^{n-1}) 
    &= \det \left(\begin{bmatrix}
        1 & \theta_1 & \theta_1^2 & \cdots & \theta_1^{n-1} \\
        1 & \theta_2 & \theta_2^2 & \cdots & \theta_2^{n-1} \\
        \cdots & \cdots & \cdots & \cdots &\cdots \\
        1 & \theta_n & \theta_n^2 & \cdots & \theta_n^{n-1}
    \end{bmatrix} \right)^2 \\
    &= \det \left(\begin{bmatrix}
        1 & \theta_1 & \theta_1^2 & \cdots & \theta_1^{n-1} \\
        0 & \theta_2 - \theta_1 & (\theta_2-\theta_1)(\theta_2 + \theta_1) & \cdots & (\theta_2-\theta_1)(\theta_2^{n-2}+\theta_2^{n-3}\theta_1+\cdots+\theta_1^{n-2}) \\
        \cdots & \cdots & \cdots & \cdots &\cdots \\
        0 & \theta_n - \theta_1 & (\theta_n-\theta_1)(\theta_n + \theta_1) & \cdots & (\theta_n-\theta_1)(\theta_n^{n-2}+\theta_n^{n-3}\theta_1+\cdots+\theta_1^{n-2})
    \end{bmatrix} \right)^2 \\
    &= \left(\prod_{i=2}^n (\theta_i - \theta_1)^2 \right) \det \left(\begin{bmatrix}
            1 & \theta_2+\theta_1 & \cdots & \theta_2^{n-2}+\theta_2^{n-3}\theta_1+\cdots+\theta_1^{n-2} \\
            \cdots & \cdots & \cdots &\cdots \\
            1 & \theta_n + \theta_1 & \cdots & \theta_n^{n-2}+\theta_n^{n-3}\theta_1+\cdots+\theta_1^{n-2}
        \end{bmatrix}\right)^2
\end{aligned}
$}}
\noindent 将最后一个式子的第一列乘以$\theta_1,\cdots\theta_1^{n-2}$并减到第2到$n-1$列,消掉所有$\theta_1^i$项,然后用第二列消掉$\theta_j\theta_1^i$次项,以此类推,实际上有

\centerline{\scalebox{0.7}{
    $\det \left(\begin{bmatrix}
            1 & \theta_2+\theta_1 & \cdots & \theta_2^{n-2}+\theta_2^{n-3}\theta_1+\cdots+\theta_1^{n-2} \\
            \cdots & \cdots & \cdots &\cdots \\
            1 & \theta_n + \theta_1 & \cdots & \theta_n^{n-2}+\theta_n^{n-3}\theta_1+\cdots+\theta_1^{n-2}
        \end{bmatrix}\right) = \det \left(\begin{bmatrix}
            1 & \theta_2 & \cdots & \theta_2^{n-2} \\
            \cdots & \cdots & \cdots &\cdots \\
            1 & \theta_n & \cdots & \theta_n^{n-2}
        \end{bmatrix}\right)
$}}
\noindent 进而可以通过第推得到
$$
d(1, \theta, \cdots, \theta^{n-1}) = \prod_{i<j}(\theta_i - \theta_j)^2
$$

\section{Extensions of Dedekind Domains}
以下的讨论中均假设$\idealo$为Dedekind domain,$K$为其分数域, $L|K$为有限代数扩张,$\idealO$为$L$中的$\idealo$的整闭包,则$\idealO$仍然是一个Dedekind domain (proof required)。考虑环$\idealo$上的素理想$p$, 其在$\idealO$中被分解为一系列素理想的乘积$p\idealO=\mathcal{P}_1^{e_1}\mathcal{P}_2^{e_2}...\mathcal{P}_r^{e_r}$,其中$e_i$被称为$\mathcal{P}$关于$p$的\textbf{指数(index)},而扩张的度数$f_i = [\idealO/\mathcal{P}_i:\idealo/p]$称为$\mathcal{P}_i$关于$p$的\textbf{度数(degree)}

如果$L|K$为一个可分域扩张,则有下述定理
\begin{theorem}
    \label{neu:th:fundamental_identity}
    $$
    \sum_{i=1}^r e_i f_i = n
    $$
    其中$n=[L:K]$
\end{theorem}

\section{Cyclotomic Fields}

设$\zeta$为一个$n$次单位根, 如果$n = \ell^v$并且$\ell$为素数, 令$\lambda = 1 - \zeta$, 则主理想$(\lambda)$是$\Q(\zeta)$的代数整数环中的素理想,该素理想的度为1(参看\ref{neu:th:fundamental_identity})并且
$$
\ell \idealo = (\lambda)^d, \text{其中}d = \phi(\ell^v) = [\Q(\zeta):\Q]
$$
此外, $1, \zeta, \cdots, \zeta^{d-1}$构成扩张$\Q(\zeta)|\Q$的一组基底, 其判别式为
$$
d(1, \zeta, \cdots, \zeta^{d-1}) = \pm\ell^s, \text{其中}s = \ell^{v-1}(v\ell - v - 1)
$$
\textbf{证明}

$\zeta$的极小多项式的根由所有形如$\zeta^g, g \in \Z / (n\Z)^*$的单位根构成(只要注意$\Q(\zeta)|\Q$的固定$\Q$的自同构群将$\zeta$映射到某个能生成所有$n$次单位根的单位根,即某个$\zeta^g$),而
$$
\Z / (n\Z) - \Z / (n\Z)^* = \{0, \ell, 2\ell, 3\ell, \cdots,  (\ell^{v-1} - 1)\ell\}
$$
$\{\zeta^h|h \in \Z / (n\Z) - \Z / (n\Z)^*\}$构成了$x^{\ell^{v-1}} -1 = 0$的所有根,于是$\zeta$的极小多项式也可以写成
$$
\begin{aligned}
    p(\zeta) 
        &= \prod_{g\in \Z/(n\Z)*}(x - \zeta^g) \\
        &= \frac{\prod_{k\in\Z/(n\Z)} (x-\zeta^k)}{\prod_{h\in \Z/(n\Z) - \Z / (n\Z)^*}{(x-\zeta^h)}} \\
        &= \frac{x^{\ell^v} - 1}{x^{\ell^{v-1}}-1} \\
        &= 1 + x^{\ell^{v-1}} + x^{2\ell^{v-1}} + x^{3\ell^{v-1}} + \cdots + x^{(\ell-1)\ell^{v-1}}
\end{aligned}
$$
上式中取$x=1$得到
$$
\prod_{g\in \Z/(n\Z)^*}(1-\zeta^g)=\ell
$$
然而$\forall g \in \Z / (n\Z)^*$,若令$g'$是$g$的逆(即$gg'=1$)则
$$
\frac{1-\zeta}{1-\zeta^g} = \frac{1-{\zeta^{g}}^{g'}}{1-\zeta^g} = 1+\zeta^g + \zeta^{2g}+ \cdots +\zeta^{(g'-1)g}
$$
反过来
$$
\frac{1-\zeta^g}{1-\zeta} = 1+\zeta + \zeta^2+ \cdots+\zeta^{g-1}
$$
说明$1-\zeta$和$1-\zeta^g$在其中只相差了一个可逆元,于是$\ell = \epsilon(1-\zeta)^{\phi(l^v)}$, 其中$\epsilon$是一个可逆元,进而$l\idealo = (\lambda)^{\phi(\ell^v)}$

\section{Another section}
\begin{equation}
    \label{example:hello}
    E = m c^2 \times 2
\end{equation}
参看公式\ref{example:hello}
\end{document}
